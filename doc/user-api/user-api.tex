\documentclass{article}
\author{s3in!c $\langle$s3inlc@hashes.org$\rangle$ }
\usepackage[T1]{fontenc}
\usepackage[utf8]{inputenc}
\usepackage[left=10mm, right=10mm, top=20mm, bottom=20mm]{geometry}
\usepackage[english]{babel}
\usepackage{fancyvrb}
\usepackage[dvipsnames]{xcolor}
\begin{document}
	\title{Hashtopolis User API (V1)}
	\maketitle
	\section*{Introduction}
	The communication for the User API is always in JSON formatted values. When sending a request to the server, it should be a POST containing the JSON data.
	Every request has a \textit{section} and a \textit{request} field to state which action should be executed. Every response again then contains the requested section and request and gives a \textit{status} to indicate if the query was successful or not. To increase the readability of this document, requests are always in blue, successful responses in green and error messages in red. \\

	The User API can be found at http[s]://[YOUR_SERVER]:8080/api/user.php. All of the below payloads should be sent to that endpoint.

	\section*{Errors}
	In case of an error with the query which the user sends to the server, the response will have following format with the corresponding action which was requested and the error message which should help in getting information about this error.
	{	
		\color{BrickRed}
		\begin{verbatim}
		{
		  "section":"task",
		  "request":"create",
		  "status":"ERROR",
		  "message":"You are not allowed to create tasks!"
		}
		\end{verbatim}
	}
	
	\section*{Sections}
	This part lists all sections available on the API. The value in the brackets denotes the according value to be sent on API queries.
	\begin{itemize}
		\item{Access Control (\textit{access})}
		\item{Agents (\textit{agent})}
		\item{Server Config (\textit{config})}
		\item{Crackers (\textit{cracker})}
		\item{Files (\textit{file})}
		\item{Groups (\textit{group})}
		\item{Hashlists (\textit{hashlist})}
		\item{Preconfigured Tasks (\textit{pretask})}
		\item{Superhashlists (\textit{superhaslist})}
		\item{Supertasks (\textit{supertask})}
		\item{Tasks (\textit{task})}
		\item{Test (\textit{test})}
		\item{User Management (\textit{user})}
	\end{itemize}
	\pagebreak

	\section*{Test (\textit{test})}
		This section is used to do testing queries, e.g. to test connectivity or availability of this API. The test section is the only one which allows to make requests without an access key.
		\subsection*{\textit{connection}}
			Used to test if the URL is a valid API endpoint.
			{
				\color{blue}
				\begin{verbatim}
				{
				  "section":"test",
				  "request":"connection"
				}
				\end{verbatim}
		    }
			
			{
				\color{OliveGreen}
				\begin{verbatim}
				{
				  "section": "test",
				  "request": "connection",
				  "response": "SUCCESS"
				}
				\end{verbatim}
			}
		\subsection*{\textit{access}}
			Used to check if a given API key is still valid and can be used.
			{
				\color{blue}
				\begin{verbatim}
				{
				  "section": "test",
				  "request": "access",
				  "accessKey": "mykey"
				}
				\end{verbatim}
			}
			{
				\color{OliveGreen}
				\begin{verbatim}
				{
				  "section": "test",
				  "request": "access",
				  "response": "OK"
				}
				\end{verbatim}
			}
			{
				\color{BrickRed}
				\begin{verbatim}
				{
				  "section": "test",
				  "request": "access",
				  "response": "ERROR",
				  "message": "API key was not found!"
				}
				\end{verbatim}
			}
	\pagebreak
		\section*{Agents (\textit{agent})}
		Used to access all functions around agents. Please note that the user must have access to the groups where an agent is member of to retrieve it and to be able to apply changes.
		\subsection*{\textit{listAgents}}
			List all agents with some basic informations.
			{
				\color{blue}
				\begin{verbatim}
				{
				  "section": "agent",
				  "request": "listAgents",
				  "accessKey": "mykey"
				}
				\end{verbatim}
			}
			{
				\color{OliveGreen}
				\begin{verbatim}
				{
				  "section": "agent",
				  "request": "listAgents",
				  "response": "OK",
				  "agents": [
				    {
				      "agentId": "2",
				      "name": "cracker1",
				      "devices": [
				        "Intel(R) Core(TM) i7-3770 CPU @ 3.40GHz",
				        "NVIDIA Quadro 600"
				      ]
				    }
				  ]
				}
				\end{verbatim}
			}
		\subsection*{\textit{get}}
			Retrieve all the informations about a specific agent by providing its ID. The last action time is a UNIX timestamp and if the configuration on the server is set to hide the IP of the agents, the value will just be \textit{Hidden} instead of the IP.
			{
				\color{blue}
				\begin{verbatim}
				{
				  "section": "agent",
				  "request": "get",
				  "agentId": 2,
				  "accessKey": "mykey"
				}
				\end{verbatim}
			}
			{
				\color{OliveGreen}
				\begin{verbatim}
				{
				  "section": "agent",
				  "request": "get",
				  "response": "OK",
				  "name": "cracker1",
				  "devices": [
				    "Intel(R) Core(TM) i7-3770 CPU @ 3.40GHz",
				    "NVIDIA Quadro 600"
				  ],
				  "owner": {
				    "userId": 1,
				    "username": "htp"
				  },
				  "isCpuOnly": false,
				  "isTrusted": true,
				  "isActive": true,
				  "token": "0lBfAp7YQh",
				  "extraParameters": "--force",
				  "errorFlag": 2,
				  "lastActivity": {
				    "action": "getTask",
				    "time": 1531316240,
				    "ip": "127.0.0.1"
				  }
				}
				\end{verbatim}
			}
		\subsection*{\textit{setActive}}
			Set an agent active/inactive. 
			{
				\color{blue}
				\begin{verbatim}
				{
				  "section": "agent",
				  "request": "setActive",
				  "active": false,
				  "agentId": 2,
				  "accessKey": "mykey"
				}
				\end{verbatim}
			}
			{
				\color{OliveGreen}
				\begin{verbatim}
				{
				  "section": "agent",
				  "request": "setActive",
				  "response": "OK"
				}
				\end{verbatim}
			}
		\subsection*{\textit{changeOwner}}
			Either set an owner for an agent or remove the owner from it. The user can either be specified by providing the user ID or the username. If no owner should be specified, the user value must be \textit{null}.
			{
				\color{blue}
				\begin{verbatim}
				{
				  "section": "agent",
				  "request": "changeOwner",
				  "user": 1,
				  "agentId": 2,
				  "accessKey": "mykey"
				}
				\end{verbatim}
			}
			{
				\color{blue}
				\begin{verbatim}
				{
				  "section": "agent",
				  "request": "changeOwner",
				  "user": "testuser",
				  "agentId": 2,
				  "accessKey": "mykey"
				}
				\end{verbatim}
			}
			{
				\color{blue}
				\begin{verbatim}
				{
				  "section": "agent",
				  "request": "changeOwner",
				  "user": null,
				  "agentId": 2,
				  "accessKey": "mykey"
				}
				\end{verbatim}
			}
			{
				\color{OliveGreen}
				\begin{verbatim}
				{
				  "section": "agent",
				  "request": "changeOwner",
				  "response": "OK"
				}
				\end{verbatim}
			}
		\subsection*{\textit{setName}}
			Set the name of the agent.
			{
				\color{blue}
				\begin{verbatim}
				{
				  "section": "agent",
				  "request": "setName",
				  "name": "cracker1",
				  "agentId": 2,
				  "accessKey": "mykey"
				}
				\end{verbatim}
			}
			{
				\color{OliveGreen}
				\begin{verbatim}
				{
				  "section": "agent",
				  "request": "setName",
				  "response": "OK"
				}
				\end{verbatim}
			}
		\subsection*{\textit{setCpuOnly}}
			Set if an agent is CPU only or not.
			{
				\color{blue}
				\begin{verbatim}
				{
				  "section": "agent",
				  "request": "setCpuOnly",
				  "cpuOnly": true,
				  "agentId": 2,
				  "accessKey": "mykey"
				}
				\end{verbatim}
			}
			{
				\color{OliveGreen}
				\begin{verbatim}
				{
				  "section": "agent",
				  "request": "setCpuOnly",
				  "response": "OK"
				}
				\end{verbatim}
			}
		\subsection*{\textit{setExtraParams}}
			Set agent specific command line parameters for the agent which are included in the cracker command line call on the agent.
			{
				\color{blue}
				\begin{verbatim}
				{
				  "section": "agent",
				  "request": "setExtraParams",
				  "extraParameters": "-d 1,2",
				  "agentId": 2,
				  "accessKey": "mykey"
				}
				\end{verbatim}
			}
			{
				\color{OliveGreen}
				\begin{verbatim}
				{
				  "section": "agent",
				  "request": "setExtraParams",
				  "response": "OK"
				}
				\end{verbatim}
			}
		\subsection*{\textit{setErrorFlag}}
			Set how errors on the agent should be handled on the server. Following values can be given as \textit{ignoreErrors} value:
			\begin{description}
				\item[0] In case of an error, the error message gets saved on the server and the agent will be put into inactive state.
				\item[1] In case of an error, the error message gets saved on the server, but the agent will be given further chunks to work on if he requests so.
				\item[2] In case of an error, nothing will be saved on the server and the agent can continue to work and will not put into inactive state.
			\end{description}
			{
				\color{blue}
				\begin{verbatim}
				{
				  "section": "agent",
				  "request": "setErrorFlag",
				  "ignoreErrors": 0,
				  "agentId": 2,
				  "accessKey": "mykey"
				}
				\end{verbatim}
			}
			{
				\color{OliveGreen}
				\begin{verbatim}
				{
				  "section": "agent",
				  "request": "setErrorFlag",
				  "response": "OK"
				}
				\end{verbatim}
			}
		\subsection*{\textit{setTrusted}}
			Set if an agent is trusted or not.
			{
				\color{blue}
				\begin{verbatim}
				{
				  "section": "agent",
				  "request": "setTrusted",
				  "trusted": false,
				  "agentId": 2,
				  "accessKey": "mykey"
				}
				\end{verbatim}
			}
			{
				\color{OliveGreen}
				\begin{verbatim}
				{
				  "section": "agent",
				  "request": "setTrusted",
				  "response": "OK"
				}
				\end{verbatim}
			}
		
		\subsection*{\textit{listVouchers}}
			Lists all currently existing vouchers on the server which can be used to register new agents.
			{
				\color{blue}
				\begin{verbatim}
				{
				  "section": "agent",
				  "request": "listVouchers",
				  "accessKey": "mykey"
				}
				\end{verbatim}
			}
			{
				\color{OliveGreen}
				\begin{verbatim}
				{
				  "section": "agent",
				  "request": "getBinaries",
				  "response": "OK",
				  "vouchers": [
				    "sM2q6CwiPY",
				    "xkw782a3x9",
				    "2drg6Vsqor",
				    "AZyY8dK1ao"
				  ]
				}
				\end{verbatim}
			}
		\subsection*{\textit{createVoucher}}
			Create a new voucher on the server. It is optional to specify a voucher code otherwise the server will just generate a random one. The server always sends back the created voucher.
			{
				\color{blue}
				\begin{verbatim}
				{
				  "section": "agent",
				  "request": "createVoucher",
				  "voucher": "mySpecificVoucher",
				  "accessKey": "mykey"
				}
				\end{verbatim}
			}
			{
				\color{blue}
				\begin{verbatim}
				{
				  "section": "agent",
				  "request": "createVoucher",
				  "accessKey": "mykey"
				}
				\end{verbatim}
			}
			{
				\color{OliveGreen}
				\begin{verbatim}
				{
				  "section": "agent",
				  "request": "createVoucher",
				  "response": "OK",
				  "voucher": "Gjawgidkr4"
				}
				\end{verbatim}
			}
		\subsection*{\textit{deleteVoucher}}
			Delete a voucher from the server.
			{
				\color{blue}
				\begin{verbatim}
				{
				  "section": "agent",
				  "request": "deleteVoucher",
				  "voucher": "Gjawgidkr4",
				  "accessKey": "mykey"
				}
				\end{verbatim}
			}
			{
				\color{OliveGreen}
				\begin{verbatim}
				{
				  "section": "agent",
				  "request": "deleteVoucher",
				  "response": "OK"
				}
				\end{verbatim}
			}
		\subsection*{\textit{deleteAgent}}
			Delete an agent from the server.
			{
				\color{blue}
				\begin{verbatim}
				{
				  "section": "agent",
				  "request": "deleteAgent",
				  "agentId": "241", 
				  "accessKey": "mykey"
				}
				\end{verbatim}
			}
			{
				\color{OliveGreen}
				\begin{verbatim}
				{
				  "section": "agent",
				  "request": "deleteAgent",
				  "response": "OK"
				}
				\end{verbatim}
			}
		\subsection*{\textit{getBinaries}}
			Lists which agent binaries are available on the server to be used for agents.
			{
				\color{blue}
				\begin{verbatim}
				{
				  "section": "agent",
				  "request": "getBinaries",
				  "accessKey": "mykey"
				}
				\end{verbatim}
			}
			{
				\color{OliveGreen}
				\begin{verbatim}
				{
				  "section": "agent",
				  "request": "getBinaries",
				  "response": "OK",
				   "apiUrl": "http:\/\/localhost\/hashtopolis\/src\/api\/api\/server.php",
				  "binaries": [
				    {
				      "name": "csharp",
				      "os": "Windows, Linux(mono), OS X(mono)",
				      "url": "http:\/\/localhost\/hashtopolis\/src\/api\/agents.php?download=1",
				      "version": "0.52.2",
				      "filename": "hashtopolis.exe"
				    },
				    {
				      "name": "python",
				      "os": "Windows, Linux, OS X",
				      "url": "http:\/\/localhost\/hashtopolis\/src\/api\/agents.php?download=2",
				      "version": "0.1.4",
				      "filename": "hashtopolis.zip"
				    }
				  ]
				}
				\end{verbatim}
			}
	\pagebreak
		\section*{Tasks (\textit{task})}
		Used to access all functions around tasks. Please note that the user must have access to the groups where a task is belonging to, to retrieve it and to be able to apply changes.
		\subsection*{\textit{listTasks}}
			List all tasks on the server. There are two task types:
			\begin{description}
				\item[0] Normal Task
				\item[1] Supertask
			\end{description}
			In case it is set in the server configuration, for normal tasks there will be a flag 'isComplete' be set which denotes if the full keyspace of the task was covered.
			For tasks, the task Id is returned, for supertasks the taskwrapper Id is returned.
			{
				\color{blue}
				\begin{verbatim}
				{
				  "section": "task",
				  "request": "listTasks",
				  "accessKey": "mykey"
				}
				\end{verbatim}
			}
			{
				\color{OliveGreen}
				\begin{verbatim}
				{
				  "section": "task",
				  "request": "listTasks",
				  "response": "OK",
				  "tasks": [
				    {
				      "taskId": 7587,
				      "name": "test 2",
				      "type": 0,
				      "hashlistId": 1,
				      "priority": 5
				    },
				    {
				      "supertaskId": 33,
				      "name": "Increment ?a",
				      "type": 1,
				      "hashlistId": 1,
				      "priority": 3
				    },
				    {
				      "supertaskId": 32,
				      "name": "Supertask Test",
				      "type": 1,
				      "hashlistId": 1,
				      "priority": 0
				    },
				    {
				      "taskId": 7580,
				      "name": "test 1",
				      "type": 0,
				      "hashlistId": 1,
				      "priority": 0
				    }
				  ]
				}
				\end{verbatim}
			}
		\subsection*{\textit{getTask}}
			Get the details for a specific task. Note that this request can only be done with tasks or subtasks, but not with supertasks.
			{
				\color{blue}
				\begin{verbatim}
				{
				  "section": "task",
				  "request": "getTask",
				  "taskId": 7587,
				  "accessKey": "mykey"
				}
				\end{verbatim}
			}
			{
				\color{OliveGreen}
				\begin{verbatim}
				{
				  "section": "task",
				  "request": "getTask",
				  "response": "OK",
				  "taskId": 7587,
				  "name": "testing",
				  "attack": "#HL# -a 0 top10000.txt -r dive.rule",
				  "chunksize": 600,
				  "color": null,
				  "benchmarkType": "speed",
				  "statusTimer": 5,
				  "priority": 0,
				  "maxAgents": 4,
				  "isCpuOnly": false,
				  "isSmall": false,
				  "skipKeyspace": 0,
				  "keyspace": 10000,
				  "dispatched": 10000,
				  "hashlistId": 1,
				  "imageUrl": "http:\/\/localhost\/hashtopolis\/src\/api\/taskimg.php?task=7587",
				  "files": [
				    {
				      "fileId": 2,
				      "filename": "dive.rule",
				      "size": 887155
				    },
				    {
				      "fileId": 3653,
				      "filename": "top10000.txt",
				      "size": 76508
				    }
				  ],
				  "speed": 0,
				  "searched": 10000,
				  "chunkIds": [
				    31
				  ],
				  "agents": [
				    {
				      "agentId": 2,
				      "benchmark": "0",
				      "speed": 0
				    }
				  ],
				  "isComplete": false,
				  "usePreprocessor": false,
				  "preprocessorId": 0,
				  "preprocessorCommand": ""
				}
				\end{verbatim}
			}
		\subsection*{\textit{listSubtasks}}
			List all subtasks of a given running supertask.
			{
				\color{blue}
				\begin{verbatim}
				{
				  "section": "task",
				  "request": "listSubtasks",
				  "supertaskId": 33,
				  "accessKey": "mykey"
				}
				\end{verbatim}
			}
			{
				\color{OliveGreen}
				\begin{verbatim}
				{
				  "section": "task",
				  "request": "listSubtasks",
				  "response": "OK",
				  "subtasks": [
				    {
				      "taskId": 7582,
				      "name": "?a?a?a",
				      "priority": 0
				    },
				    {
				      "taskId": 7583,
				      "name": "?a?a?a?a",
				      "priority": 0
				    },
				    {
				      "taskId": 7584,
				      "name": "?a?a?a?a?a",
				      "priority": 0
				    },
				    {
				      "taskId": 7585,
				      "name": "?a?a?a?a?a?a",
				      "priority": 0
				    }
				  ]
				}
				\end{verbatim}
			}
		\subsection*{\textit{getChunk}}
			Get details about a specific chunk. Progress is given in percents, start/length/checkpoint are in relation to the keyspace.
			{
				\color{blue}
				\begin{verbatim}
				{
				  "section": "task",
				  "request": "getChunk",
				  "chunkId": 30,
				  "accessKey": "mykey"
				}
				\end{verbatim}
			}
			{
				\color{OliveGreen}
				\begin{verbatim}
				{
				  "section": "task",
				  "request": "getChunk",
				  "response": "OK",
				  "chunkId": 30,
				  "start": 23141360,
				  "length": 5785340,
				  "checkpoint": 28926700,
				  "progress": 100,
				  "taskId": 7585,
				  "agentId": 2,
				  "dispatchTime": 1531313146,
				  "lastActivity": 1531313738,
				  "state": 4,
				  "cracked": 0,
				  "speed": 0
				}
				\end{verbatim}
			}
			\subsection*{\textit{createTask}}
				Create a new task (one example with files and one without).
				{
					\color{blue}
					\begin{verbatim}
					{
					  "section": "task",
					  "request": "createTask",
					  "name": "API Task",
					  "hashlistId": 1,
					  "attackCmd": "#HL# -a 0 -r dive.rule example.dict",
					  "chunksize": 600,
					  "statusTimer": 5,
					  "benchmarkType": "speed",
					  "color": "5D5D5D",
					  "isCpuOnly": false,
					  "isSmall": false,
					  "skip": 0,
					  "crackerVersionId": 2,
					  "files": [
					    1,
					    2
					  ],
					  "priority": 100,
					  "maxAgents": 4,
					  "preprocessorId": 0,
					  "preprocessorCommand": "",
					  "accessKey": "mykey"
					}
					\end{verbatim}
				}
				{
					\color{blue}
					\begin{verbatim}
					{
					  "section": "task",
					  "request": "createTask",
					  "name": "API Task BF",
					  "hashlistId": 1,
					  "attackCmd": "#HL# -a 3 ?l?l?l?l?l?l",
					  "chunksize": 600,
					  "statusTimer": 5,
					  "benchmarkType": "speed",
					  "color": "5D5D5D",
					  "isCpuOnly": false,
					  "isSmall": true,
					  "skip": 0,
					  "crackerVersionId": 2,
					  "files": [],
					  "priority": 99,
					  "maxAgents": 4,
					  "preprocessorId": 0,
					  "preprocessorCommand": "",
					  "accessKey": "mykey"
					}
					\end{verbatim}
				}
				{
					\color{OliveGreen}
					\begin{verbatim}
					{
					  "section": "task",
					  "request": "createTask",
					  "response": "OK",
					  "taskId": 101
					}
					\end{verbatim}
				}
		\subsection*{\textit{runPretask}}
			Create a task based on a preconfigured task.
			{
				\color{blue}
				\begin{verbatim}
				{
				  "section": "task",
				  "request": "runPretask",
				  "name": "API Run pretask",
				  "hashlistId": 1,
				  "pretaskId": 1,
				  "crackerVersionId": 2,
				  "accessKey": "mykey"
				}
				\end{verbatim}
			}
			{
				\color{OliveGreen}
				\begin{verbatim}
				{
				  "section": "task",
				  "request": "runPretask",
				  "response": "OK"
				}
				\end{verbatim}
			}		
		\subsection*{\textit{runSupertask}}
			Create a supertask out of a configured preconfigured task collection.
			{
				\color{blue}
				\begin{verbatim}
				{
				  "section": "task",
				  "request": "runSupertask",
				  "hashlistId": 1,
				  "supertaskId": 1,
				  "crackerVersionId": 2,
				  "accessKey": "mykey"
				}
				\end{verbatim}
			}
			{
				\color{OliveGreen}
				\begin{verbatim}
				{
				  "section": "task",
				  "request": "runSupertask",
				  "response": "OK"
				}
				\end{verbatim}
			}	
		\subsection*{\textit{setTaskPriority}}
			Set the priority for a task.
			{
				\color{blue}
				\begin{verbatim}
				{
				  "section": "task",
				  "request": "setTaskPriority",
				  "taskId": 7580,
				  "priority": 9000,
				  "accessKey": "mykey"
				}
				\end{verbatim}
			}
			{
				\color{OliveGreen}
				\begin{verbatim}
				{
				  "section": "task",
				  "request": "setTaskPriority",
				  "response": "OK"
				}
				\end{verbatim}
			}		
		\subsection*{\textit{setSupertaskPriority}}
			Set the priority for a supertask.
			The required supertaskId parameter corresponds to the taskwrapper Id of the running supertask.
			Note that the required taskwrapper Id can be queried by the API call listTasks.
			{
				\color{blue}
				\begin{verbatim}
				{
				  "section": "task",
				  "request": "setSupertaskPriority",
				  "supertaskId": 42,
				  "supertaskPriority": 9000,
				  "accessKey": "mykey"
				}
				\end{verbatim}
			}
			{
				\color{OliveGreen}
				\begin{verbatim}
				{
				  "section": "task",
				  "request": "setSupertaskPriority",
				  "response": "OK"
				}
				\end{verbatim}
			}
		\subsection*{\textit{setTaskMaxAgents}}
			Set the maximum number of assigned agents for a task. Takes effect when agents request a new chunk.
			{
				\color{blue}
				\begin{verbatim}
				{
				  "section": "task",
				  "request": "setTaskMaxAgents",
				  "taskId": 7580,
				  "maxAgents": 2,
				  "accessKey": "mykey"
				}
				\end{verbatim}
			}
			{
				\color{OliveGreen}
				\begin{verbatim}
				{
				  "section": "task",
				  "request": "setTaskMaxAgents",
				  "response": "OK"
				}
				\end{verbatim}
			}
		\subsection*{\textit{setSupertaskMaxAgents}}
			Set the maximum number of assigned agents for a running supertask. Takes effect when agents request a new chunk.
			{
				\color{blue}
				\begin{verbatim}
				{
				  "section": "task",
				  "request": "setSupertaskMaxAgents",
				  "supertaskId": 42,
				  "supertaskMaxAgents": 2,
				  "accessKey": "mykey"
				}
				\end{verbatim}
			}
			{
				\color{OliveGreen}
				\begin{verbatim}
				{
				  "section": "task",
				  "request": "setSupertaskMaxAgents",
				  "response": "OK"
				}
				\end{verbatim}
			}
		\subsection*{\textit{setTaskName}}
			Set the name for a task.
			{
				\color{blue}
				\begin{verbatim}
				{
				  "section": "task",
				  "request": "setTaskName",
				  "taskId": 7580,
				  "name": "New Task Name",
				  "accessKey": "mykey"
				}
				\end{verbatim}
			}
			{
				\color{OliveGreen}
				\begin{verbatim}
				{
				  "section": "task",
				  "request": "setTaskName",
				  "response": "OK"
				}
				\end{verbatim}
			}
		\subsection*{\textit{setTaskColor}}
			Set the color of a task.
			{
				\color{blue}
				\begin{verbatim}
				{
				  "section": "task",
				  "request": "setTaskColor",
				  "taskId": 7580,
				  "color": "78ABCD",
				  "accessKey": "mykey"
				}
				\end{verbatim}
			}
			{
				\color{OliveGreen}
				\begin{verbatim}
				{
				  "section": "task",
				  "request": "setTaskColor",
				  "response": "OK"
				}
				\end{verbatim}
			}
		\subsection*{\textit{setTaskCpuOnly}}
			Set if a task is a CPU only task or not.
			{
				\color{blue}
				\begin{verbatim}
				{
				  "section": "task",
				  "request": "setTaskCpuOnly",
				  "taskId": 7580,
				  "isCpuOnly": false,
				  "accessKey": "mykey"
				}
				\end{verbatim}
			}
			{
				\color{OliveGreen}
				\begin{verbatim}
				{
				  "section": "task",
				  "request": "setTaskCpuOnly",
				  "response": "OK"
				}
				\end{verbatim}
			}
		\subsection*{\textit{setTaskSmall}}
			Set if a task is small or not.
			{
				\color{blue}
				\begin{verbatim}
				{
				  "section": "task",
				  "request": "setTaskSmall",
				  "taskId": 7580,
				  "isSmall": false,
				  "accessKey": "mykey"
				}
				\end{verbatim}
			}
			{
				\color{OliveGreen}
				\begin{verbatim}
				{
				  "section": "task",
				  "request": "setTaskSmall",
				  "response": "OK"
				}
				\end{verbatim}
			}
		\subsection*{\textit{taskUnassignAgent}}
			Unassign an agent from his task.
			{
				\color{blue}
				\begin{verbatim}
				{
				  "section": "task",
				  "request": "taskUnassignAgent",
				  "agentId": 2,
				  "accessKey": "mykey"
				}
				\end{verbatim}
			}
			{
				\color{OliveGreen}
				\begin{verbatim}
				{
				  "section": "task",
				  "request": "taskUnassignAgent",
				  "response": "OK"
				}
				\end{verbatim}
			}
		\subsection*{\textit{taskUnassignAgent}}
			Assign an agent to this task. Note that the agent might be re-assigned to another task if there is one with a higher priority.
			{
				\color{blue}
				\begin{verbatim}
				{
				  "section": "task",
				  "request": "taskAssignAgent",
				  "agentId": 2,
				  "taskId": 5,
				  "accessKey": "mykey"
				}
				\end{verbatim}
			}
			{
				\color{OliveGreen}
				\begin{verbatim}
				{
				  "section": "task",
				  "request": "taskAssignAgent",
				  "response": "OK"
				}
				\end{verbatim}
			}
		\subsection*{\textit{deleteTask}}
			Completely delete a task.
			{
				\color{blue}
				\begin{verbatim}
				{
				  "section": "task",
				  "request": "deleteTask",
				  "taskId": 7580,
				  "accessKey": "mykey"
				}
				\end{verbatim}
			}
			{
				\color{OliveGreen}
				\begin{verbatim}
				{
				  "section": "task",
				  "request": "deleteTask",
				  "response": "OK"
				}
				\end{verbatim}
			}		
		\subsection*{\textit{purgeTask}}
			Purge all task data and reset it to initial state.
			{
				\color{blue}
				\begin{verbatim}
				{
				  "section": "task",
				  "request": "purgeTask",
				  "taskId": 7591,
				  "accessKey": "mykey"
				}
				\end{verbatim}
			}
			{
				\color{OliveGreen}
				\begin{verbatim}
				{
				  "section": "task",
				  "request": "purgeTask",
				  "response": "OK"
				}
				\end{verbatim}
			}
		\subsection*{\textit{setSupertaskName}}
			Set the name of a running supertask.
			{
				\color{blue}
				\begin{verbatim}
				{
				  "section": "task",
				  "request": "setSupertaskName",
				  "supertaskId": 43,
				  "name": "New Supertask Name",
				  "accessKey": "mykey"
				}
				\end{verbatim}
			}
			{
				\color{OliveGreen}
				\begin{verbatim}
				{
				  "section": "task",
				  "request": "setSupertaskName",
				  "response": "OK"
				}
				\end{verbatim}
			}
		\subsection*{\textit{deleteSupertask}}
			Delete a running supertask. This includes all contained subtasks.
			{
				\color{blue}
				\begin{verbatim}
				{
				  "section": "task",
				  "request": "deleteSupertask",
				  "supertaskId": 43,
				  "accessKey": "mykey"
				}
				\end{verbatim}
			}
			{
				\color{OliveGreen}
				\begin{verbatim}
				{
				  "section": "task",
				  "request": "deleteSupertask",
				  "response": "OK"
				}
				\end{verbatim}
			}
\subsection*{\textit{archiveTask}}
	Archive a task.
	{
		\color{blue}
		\begin{verbatim}
		{
		  "section": "task",
		  "request": "archiveTask",
		  "taskId": 7601,
		  "accessKey": "mykey"
		}
		\end{verbatim}
	}
	{
		\color{OliveGreen}
		\begin{verbatim}
		{
		  "section": "task",
		  "request": "archiveTask",
		  "response": "OK"
		}
		\end{verbatim}
	}
\subsection*{\textit{archiveSupertask}}
	Archive a supertask (including all subtasks).
	{
		\color{blue}
		\begin{verbatim}
		{
		  "section": "task",
		  "request": "archiveSupertask",
		  "supertaskId": 54,
		  "accessKey": "mykey"
		}
		\end{verbatim}
	}
	{
		\color{OliveGreen}
		\begin{verbatim}
		{
		  "section": "task",
		  "request": "archiveSupertask",
		  "response": "OK"
		}
		\end{verbatim}
	}

\subsection*{\textit{getCracked}}
	Retrieve all cracked hashes by a given task.
	{
		\color{blue}
		\begin{verbatim}
		{
		  "section": "task",
		  "request": "getCracked",
		  "taskId": 100,
		  "accessKey": "mykey"
		}
		\end{verbatim}
	}
	{
		\color{OliveGreen}
		\begin{verbatim}
		{
		  "section": "task",
		  "request": "getCracked",
		  "response": "OK",
		  "cracked": [
		    {
		      "hash": "098f6bcd4621d373cade4e832627b4f6",
		      "plain": "test",
		      "crackpos": "634721"
		    },
		    {
		      "hash": "5f4dcc3b5aa765d61d8327deb882cf99",
		      "plain": "password",
		      "crackpos": "608529"
		    }
		  ]
		}
		\end{verbatim}
	}

\subsection*{\textit{setTaskTopPriority}}
	The server sets a task to top priority.
	{
		\color{blue}
		\begin{verbatim}
		{
		  "section": "task",
		  "request": "setTaskTopPriority",
		  "taskId": "1",
		  "accessKey": "myKey"
		}
		\end{verbatim}
	}
	{
		\color{OliveGreen}
		\begin{verbatim}
		{
		  "section": "task",
		  "request": "setTaskTopPriority",
		  "response": "OK"
		}
		\end{verbatim}
	}















	\pagebreak
	\section*{Preconfigured Tasks (\textit{pretask})}
		Used to access all functions around preconfigured tasks. 
		\subsection*{\textit{setPretaskPriority}}
			Set the priority of a preconfigured task.
			{
				\color{blue}
				\begin{verbatim}
				{
				  "section": "pretask",
				  "request": "setPretaskPriority",
				  "pretaskId": 1,
				  "priority": 100,
				  "accessKey": "mykey"
				}
				\end{verbatim}
			}
			{
				\color{OliveGreen}
				\begin{verbatim}
				{
				  "section": "pretask",
				  "request": "setPretaskPriority",
				  "response": "OK"
				}
				\end{verbatim}
			}
		\subsection*{\textit{setPretaskMaxAgents}}
			Set the maximum number of assigned agents for a pretask.
			{
				\color{blue}
				\begin{verbatim}
				{
				  "section": "pretask",
				  "request": "setPretaskMaxAgents",
				  "pretaskId": 1,
				  "maxAgents": 2,
				  "accessKey": "mykey"
				}
				\end{verbatim}
			}
			{
				\color{OliveGreen}
				\begin{verbatim}
				{
				  "section": "pretask",
				  "request": "setPretaskMaxAgents",
				  "response": "OK"
				}
				\end{verbatim}
			}
		\subsection*{\textit{setPretaskName}}
			Rename a preconfigured task.
			{
				\color{blue}
				\begin{verbatim}
				{
				  "section": "pretask",
				  "request": "setPretaskName",
				  "pretaskId": 1,
				  "name": "New name",
				  "accessKey": "mykey"
				}
				\end{verbatim}
			}
			{
				\color{OliveGreen}
				\begin{verbatim}
				{
				  "section": "pretask",
				  "request": "setPretaskName",
				  "response": "OK"
				}
				\end{verbatim}
			}
		\subsection*{\textit{setPretaskColor}}
			Set the color of a preconfigured task.
			{
				\color{blue}
				\begin{verbatim}
				{
				  "section": "pretask",
				  "request": "setPretaskColor",
				  "pretaskId": 1,
				  "color": "FF00FF",
				  "accessKey": "mykey"
				}
				\end{verbatim}
			}
			{
				\color{OliveGreen}
				\begin{verbatim}
				{
				  "section": "pretask",
				  "request": "setPretaskColor",
				  "response": "OK"
				}
				\end{verbatim}
			}
		\subsection*{\textit{setPretaskChunksize}}
			Set the chunk time for a preconfigured task.
			{
				\color{blue}
				\begin{verbatim}
				{
				  "section": "pretask",
				  "request": "setPretaskChunksize",
				  "pretaskId": 1,
				  "chunksize": 300,
				  "accessKey": "mykey"
				}
				\end{verbatim}
			}
			{
				\color{OliveGreen}
				\begin{verbatim}
				{
				  "section": "pretask",
				  "request": "setPretaskChunksize",
				  "response": "OK"
				}
				\end{verbatim}
			}
		\subsection*{\textit{setPretaskCpuOnly}}
			Set if a preconfigured task is a CPU only task or not.
			{
				\color{blue}
				\begin{verbatim}
				{
				  "section": "pretask",
				  "request": "setPretaskCpuOnly",
				  "pretaskId": 1,
				  "isCpuOnly": true,
				  "accessKey": "mykey"
				}
				\end{verbatim}
			}
			{
				\color{OliveGreen}
				\begin{verbatim}
				{
				  "section": "pretask",
				  "request": "setPretaskCpuOnly",
				  "response": "OK"
				}
				\end{verbatim}
			}
		\subsection*{\textit{setPretaskSmall}}
			Set if a preconfigured task is small or not.
			{
				\color{blue}
				\begin{verbatim}
				{
				  "section": "pretask",
				  "request": "setPretaskSmall",
				  "pretaskId": 1,
				  "isSmall": true,
				  "accessKey": "mykey"
				}
				\end{verbatim}
			}
			{
				\color{OliveGreen}
				\begin{verbatim}
				{
				  "section": "pretask",
				  "request": "setPretaskSmall",
				  "response": "OK"
				}
				\end{verbatim}
			}
		\subsection*{\textit{deletePretask}}
			Deletes a preconfigured task. This also includes removing it from supertasks if it is used there.
			{
				\color{blue}
				\begin{verbatim}
				{
				  "section": "pretask",
				  "request": "deletePretask",
				  "pretaskId": 1,
				  "accessKey": "mykey"
				}
				\end{verbatim}
			}
			{
				\color{OliveGreen}
				\begin{verbatim}
				{
				  "section": "pretask",
				  "request": "deletePretask",
				  "response": "OK"
				}
				\end{verbatim}
			}
		\subsection*{\textit{createPretask}}
			Create a new preconfigured task.
			{
				\color{blue}
				\begin{verbatim}
				{
				  "section": "pretask",
				  "request": "createPretask",
				  "name": "API Pretask",
				  "attackCmd": "#HL# -a 0 example.dict",
				  "chunksize": 600,
				  "statusTimer": 5,
				  "benchmarkType": "speed",
				  "color": "",
				  "isCpuOnly": false,
				  "isSmall": true,
				  "priority": 9000,
				  "maxAgents": 4,
				  "files": [
				    1
				  ],
				  "crackerTypeId": 1,
				  "accessKey": "mykey"
				}
				\end{verbatim}
			}
			{
				\color{OliveGreen}
				\begin{verbatim}
				{
				  "section": "pretask",
				  "request": "createPretask",
				  "response": "OK"
				}
				\end{verbatim}
			}
		\subsection*{\textit{listPretasks}}
			Lists all preconfigured tasks available on the server.
			{
				\color{blue}
				\begin{verbatim}
				{
				  "section": "pretask",
				  "request": "listPretasks",
				  "accessKey": "mykey"
				}
				\end{verbatim}
			}
			{
				\color{OliveGreen}
				\begin{verbatim}
				{
				  "section": "pretask",
				  "request": "listPretasks",
				  "response": "OK",
				  "pretasks": [
				    {
				      "pretaskId": 1,
				      "name": "Test Pre",
				      "priority": 0
				    },
				    {
				      "pretaskId": 7,
				      "name": "Brute ?a 8",
				      "priority": 0
				    }
				  ]
				}
				\end{verbatim}
			}
	\subsection*{\textit{getPretask}}
		Get complete information about a preconfigured task.
			{
				\color{blue}
				\begin{verbatim}
				{
				  "section": "pretask",
				  "request": "getPretask",
				  "pretaskId": 7,
				  "accessKey": "mykey"
				}
				\end{verbatim}
			}
			{
				\color{OliveGreen}
				\begin{verbatim}
				{
				  "section": "pretask",
				  "request": "getPretask",
				  "response": "OK",
				  "pretaskId": 7,
				  "name": "Brute ?a 8",
				  "attackCmd": "#HL# -a 3 ?a?a?a?a?a?a?a?a",
				  "chunksize": 600,
				  "color": null,
				  "benchmarkType": "runtime",
				  "statusTimer": 5,
				  "priority": 0,
				  "maxAgents": 4,
				  "isCpuOnly": false,
				  "isSmall": false,
				  "files": []
				}
				\end{verbatim}
			}
	\pagebreak
	\section*{Supertasks (\textit{supertask})}
	Used to access all functions around prepared supertasks. 
	\subsection*{\textit{importSupertask}}
		Create a supertask configuration with a given list of masks.
		{
			\color{blue}
			\begin{verbatim}
			{
			  "section": "supertask",
			  "request": "importSupertask",
			  "name": "Mask Supertask",
			  "maxAgents": 10,
			  "isCpuOnly": false,
			  "isSmall": false,
			  "masks": [
			    "?d?d?d?d",
			    "?l?d?d?d?d",
			    "?d?d?d?d?d",
			    "?u?d?d?d?d"
			  ],
			  "optimizedFlag": true,
			  "crackerTypeId": 1,
			  "benchmarkType": "speed",
			  "accessKey": "mykey"
			}
			\end{verbatim}
		}
		{
			\color{OliveGreen}
			\begin{verbatim}
			{
			  "section": "supertask",
			  "request": "importSupertask",
			  "response": "OK"
			}
			\end{verbatim}
		}	
	\subsection*{\textit{listSupertasks}}
		Lists the available supertasks on the server which group preconfigured tasks together.
		{
			\color{blue}
			\begin{verbatim}
			{
			  "section": "supertask",
			  "request": "listSupertasks",
			  "accessKey": "mykey"
			}
			\end{verbatim}
		}
		{
			\color{OliveGreen}
			\begin{verbatim}
			{
			  "section": "supertask",
			  "request": "listSupertasks",
			  "response": "OK",
			  "supertasks": [
			    {
			      "supertaskId": 1,
			      "name": "Supertask Test"
			    },
			    {
			      "supertaskId": 2,
			      "name": "Increment ?a"
			    }
			  ]
			}
			\end{verbatim}
		}
	\subsection*{\textit{getSupertask}}
		Get detail information of a supertask.
		{
			\color{blue}
			\begin{verbatim}
			{
			  "section": "supertask",
			  "request": "getSupertask",
			  "supertaskId": 2,
			  "accessKey": "mykey"
			}
			\end{verbatim}
		}
		{
			\color{OliveGreen}
			\begin{verbatim}
			{
			  "section": "supertask",
			  "request": "getSupertask",
			  "response": "OK",
			  "supertaskId": 2,
			  "name": "Increment ?a",
			  "pretasks": [
			    {
			      "pretaskId": 2,
			      "name": "?a?a?a",
			      "priority": 6
			    },
			    {
			      "pretaskId": 3,
			      "name": "?a?a?a?a",
			      "priority": 5
			    },
			    {
			      "pretaskId": 4,
			      "name": "?a?a?a?a?a",
			      "priority": 4
			    },
			    {
			      "pretaskId": 5,
			      "name": "?a?a?a?a?a?a",
			      "priority": 3
			    },
			    {
			      "pretaskId": 6,
			      "name": "?a?a?a?a?a?a?a",
			      "priority": 2
			    },
			    {
			      "pretaskId": 1,
			      "name": "Test Pre",
			      "priority": 0
			    }
			  ]
			}
			\end{verbatim}
		}
\subsection*{\textit{createSupertask}}
	Create a new supertask out of existing preconfigured tasks.
	{
		\color{blue}
		\begin{verbatim}
		{
		  "section": "supertask",
		  "request": "createSupertask",
		  "name": "Mixed Supertask",
		  "pretasks": [
		    7,
		    8
		  ],
		  "accessKey": "mykey"
		}
		\end{verbatim}
	}
	{
		\color{OliveGreen}
		\begin{verbatim}
		{
		  "section": "supertask",
		  "request": "createSupertask",
		  "response": "OK"
		}
		\end{verbatim}
	}
\subsection*{\textit{setSupertaskName}}
	Rename a supertask to a new name.
	{
		\color{blue}
		\begin{verbatim}
		{
		  "section": "supertask",
		  "request": "setSupertaskName",
		  "supertaskId": 4,
		  "name": "Other Name",
		  "accessKey": "mykey"
		}
		\end{verbatim}
	}
	{
		\color{OliveGreen}
		\begin{verbatim}
		{
		  "section": "supertask",
		  "request": "setSupertaskName",
		  "response": "OK"
		}
		\end{verbatim}
	}
\subsection*{\textit{deleteSupertask}}
	Delete a supertask.
	{
		\color{blue}
		\begin{verbatim}
		{
		  "section": "supertask",
		  "request": "deleteSupertask",
		  "supertaskId": 4,
		  "accessKey": "mykey"
		}
		\end{verbatim}
	}
	{
		\color{OliveGreen}
		\begin{verbatim}
		{
		  "section": "supertask",
		  "request": "deleteSupertask",
		  "response": "OK"
		}
		\end{verbatim}
	}
\subsection*{\textit{bulkSupertask}}
	Create a supertask with a base command and replace the FILE placeholder with every file it should iterate over.
	{
		\color{blue}
		\begin{verbatim}
		{
		  "section": "supertask",
		  "request": "bulkSupertask",
		  "name": "User API Bulk Test",
		  "isCpuOnly": false,
		  "isSmall": false,
		  "crackerTypeId": 1,
		  "benchmarkType": "speed",
		  "attackCmd": "#HL# example.dict -r FILE",
		  "basefiles": [
		    1
		  ],
		  "iterfiles": [
		    2,
		    3
		  ],
		  "accessKey": "mykey"
		}
		\end{verbatim}
	}
	{
		\color{OliveGreen}
		\begin{verbatim}
		{
		  "section": "supertask",
		  "request": "bulkSupertask",
		  "response": "OK"
		}
		\end{verbatim}
	}
\subsection*{\textit{setSupertaskTopPriority}}
	The server sets a supertask to top priority.
	{
		\color{blue}
		\begin{verbatim}
		{
		  "section": "task",
		  "request": "setSupertaskTopPriority",
		  "supertaskId": "1",
		  "accessKey": "myKey"
		}
		\end{verbatim}
	}
	{
		\color{OliveGreen}
		\begin{verbatim}
		{
		  "section": "task",
		  "request": "setSupertaskTopPriority",
		  "response": "OK"
		}
		\end{verbatim}
	}



	
	\pagebreak
	\section*{Hashlists (\textit{hashlist})}
	Used to access all functions around hashlists.

\subsection*{\textit{listsHashlists}}
	List all hashlists (excluding superhashlists);
	{
		\color{blue}
		\begin{verbatim}
		{
		  "section": "hashlist",
		  "request": "listHashlists",
		  "accessKey": "mykey"
		}
		\end{verbatim}
	}
	{
		\color{OliveGreen}
		\begin{verbatim}
		{
		  "section": "hashlist",
		  "request": "listHashlists",
		  "response": "OK",
		  "hashlists": [
		    {
		      "hashlistId": 1,
		      "hashtypeId": 0,
		      "name": "Hashcat Example",
		      "format": 0,
		      "hashCount": 6494
		    },
		    {
		      "hashlistId": 3,
		      "hashtypeId": 14800,
		      "name": "iTunes test for splitting",
		      "format": 0,
		      "hashCount": 1
		    },
		    {
		      "hashlistId": 4,
		      "hashtypeId": 6242,
		      "name": "truecrypt test",
		      "format": 2,
		      "hashCount": 1
		    }
		  ]
		}
		\end{verbatim}
	}
\subsection*{\textit{getHashlist}}
	Get information about a specific hashlist.
	{
		\color{blue}
		\begin{verbatim}
		{
		  "section": "hashlist",
		  "request": "getHashlist",
		  "hashlistId": 1,
		  "accessKey": "mykey"
		}
		\end{verbatim}
	}
	{
		\color{OliveGreen}
		\begin{verbatim}
		{
		  "section": "hashlist",
		  "request": "getHashlist",
		  "response": "OK",
		  "hashlistId": 1,
		  "hashtypeId": 0,
		  "name": "Hashcat Example",
		  "format": 0,
		  "hashCount": 6494,
		  "cracked": 3382,
		  "accessGroupId": 1,
		  "isHexSalt": false,
		  "isSalted": false,
		  "isSecret": false,
		  "saltSeparator": ":",
		  "notes": "This hashlist is from blahblah...",
		  "useBrain": false
		}
		\end{verbatim}
	}
\subsection*{\textit{createHashlist}}
	Create a new hashlist. Please note that it is not ideal to create large hashlists with the API as you have to send the full data. The hashlist data should always be base64 (using UTF-8) encoded. Hashcat brain can only be used if it is activated in the server config.
	{
		\color{blue}
		\begin{verbatim}
		{
		  "section": "hashlist",
		  "request": "createHashlist",
		  "name": "API Hashlist",
		  "isSalted": false,
		  "isSecret": true,
		  "isHexSalt": false,
		  "separator": ":",
		  "format": 0,
		  "hashtypeId": 3200,
		  "accessGroupId": 1,
		  "data": "JDJ5JDEyJDcwMElMNlZ4TGwyLkEvS2NISmJEYmVKMGFhcWVxYUdrcHhlc0FFZC5jWFBQUU4vWjNVN1c2",
		  "useBrain": false,
		  "brainFeatures": 0,
		  "accessKey": "mykey"
		}
		\end{verbatim}
	}
	{
		\color{OliveGreen}
		\begin{verbatim}
		{
		  "section": "hashlist",
		  "request": "createHashlist",
		  "response": "OK",
		  "hashlistId": 101
		}
		\end{verbatim}
	}
\subsection*{\textit{setHashlistName}}
	Set the name of a hashlist.
	{
		\color{blue}
		\begin{verbatim}
		{
		  "section": "hashlist",
		  "request": "setHashlistName",
		  "name": "BCRYPT easy",
		  "hashlistId": 5,
		  "accessKey": "mykey"
		}
		\end{verbatim}
	}
	{
		\color{OliveGreen}
		\begin{verbatim}
		{
		  "section": "hashlist",
		  "request": "setHashlistName",
		  "response": "OK"
		}
		\end{verbatim}
	}
\subsection*{\textit{setSecret}}
	Set if a hashlist is secret or not.
	{
		\color{blue}
		\begin{verbatim}
		{
		  "section": "hashlist",
		  "request": "setSecret",
		  "isSecret": false,
		  "hashlistId": 5,
		  "accessKey": "mykey"
		}
		\end{verbatim}
	}
	{
		\color{OliveGreen}
		\begin{verbatim}
		{
		  "section": "hashlist",
		  "request": "setSecret",
		  "response": "OK"
		}
		\end{verbatim}
	}
\subsection*{\textit{importCracked}}
	Add some cracked hashes from an external source for this hashlist. The data must be base64 (using UTF-8) encoded.
	{
		\color{blue}
		\begin{verbatim}
		{
		  "section": "hashlist",
		  "request": "importCracked",
		  "hashlistId": 5,
		  "separator": ":",
		  "data": "JDJ5JDEyJDcwMElMNlZ4TGwyLkEvS2NISmJEYmVKMGFhcWVxYUdrcHhlc0FFZC5jWFBQUU4vWjNVN1c2OnRlc3Q=",
		  "accessKey": "mykey"
		}
		\end{verbatim}
	}
	{
		\color{OliveGreen}
		\begin{verbatim}
		{
		  "section": "hashlist",
		  "request": "importCracked",
		  "response": "OK",
		  "linesProcessed": 1,
		  "newCracked": 1,
		  "alreadyCracked": 0,
		  "invalidLines": 0,
		  "notFound": 0,
		  "processTime": 0,
		  "tooLongPlains": 0
		}
		\end{verbatim}
	}
\subsection*{\textit{exportCracked}}
	Exports the cracked hashes in hash:plain format to a new file. The response includes the informations about the created file.
	{
		\color{blue}
		\begin{verbatim}
		{
		  "section": "hashlist",
		  "request": "exportCracked",
		  "hashlistId": 5,
		  "accessKey": "mykey"
		}
		\end{verbatim}
	}
	{
		\color{OliveGreen}
		\begin{verbatim}
		{
		  "section": "hashlist",
		  "request": "exportCracked",
		  "response": "OK",
		  "fileId": 7567,
		  "filename": "Pre-cracked_5_19-07-2018_14-45-52.txt"
		}
		\end{verbatim}
	}
\subsection*{\textit{generateWordlist}}
	Generates a wordlist of all plaintexts of the cracked hashes of this hashlist. The response includes the informations about the created file.
	{
		\color{blue}
		\begin{verbatim}
		{
		  "section": "hashlist",
		  "request": "generateWordlist",
		  "hashlistId": 5,
		  "accessKey": "mykey"
		}
		\end{verbatim}
	}
	{
		\color{OliveGreen}
		\begin{verbatim}
		{
		  "section": "hashlist",
		  "request": "generateWordlist",
		  "response": "OK",
		  "fileId": 7568,
		  "filename": "Wordlist_5_19.07.2018_14.47.20.txt"
		}
		\end{verbatim}
	}
\subsection*{\textit{exportLeft}}
	Generates a left list with all hashes which are not cracked. The response returns informations about the created file. This only works for plaintext hashlists!
	{
		\color{blue}
		\begin{verbatim}
		{
		  "section": "hashlist",
		  "request": "exportLeft",
		  "hashlistId": 1,
		  "accessKey": "mykey"
		}
		\end{verbatim}
	}
	{
		\color{OliveGreen}
		\begin{verbatim}
		{
		  "section": "hashlist",
		  "request": "exportLeft",
		  "response": "OK",
		  "fileId": 7569,
		  "filename": "Leftlist_1_19-07-2018_14-49-02.txt"
		}
		\end{verbatim}
	}
\subsection*{\textit{deleteHashlist}}
	Delete a hashlist and all according hashes. This will remove a hashlist from the superhashlists it is member of.
	{
		\color{blue}
		\begin{verbatim}
		{
		  "section": "hashlist",
		  "request": "deleteHashlist",
		  "hashlistId": 5,
		  "accessKey": "mykey"
		}
		\end{verbatim}
	}
	{
		\color{OliveGreen}
		\begin{verbatim}
		{
		  "section": "hashlist",
		  "request": "deleteHashlist",
		  "response": "OK"
		}
		\end{verbatim}
	}
\subsection*{\textit{getHash}}
	Search if a hash is found on the server. This searches on all hashlists which the user has access to.
	{
		\color{blue}
		\begin{verbatim}
		{
		  "section": "hashlist",
		  "request": "getHash",
		  "hash": "0021ca52049c734ac0d3d6f92042abf7",
		  "accessKey": "mykey"
		}
		\end{verbatim}
	}
	{
		\color{OliveGreen}
		\begin{verbatim}
		{
		  "section": "hashlist",
		  "request": "getHash",
		  "response": "ERROR",
		  "message": "Hash was not found or is not cracked!"
		}
		\end{verbatim}
	}
	{
		\color{blue}
		\begin{verbatim}
		{
		  "section": "hashlist",
		  "request": "getHash",
		  "hash": "00428d94d9482d8c7037b6865521b3fd",
		  "accessKey": "mykey"
		}
		\end{verbatim}
	}
	{
		\color{OliveGreen}
		\begin{verbatim}
		{
		  "section": "hashlist",
		  "request": "getHash",
		  "response": "OK",
		  "hash": "00428d94d9482d8c7037b6865521b3fd",
		  "crackpos": 12467,
		  "plain": "wellgetthem"
		}
		\end{verbatim}
	}
\subsection*{\textit{getCracked}}
	Retrieve all cracked hashes of a given hashlist.
	{
		\color{blue}
		\begin{verbatim}
		{
		  "section": "hashlist",
		  "request": "getCracked",
		  "hashlistId": "1",
		  "accessKey": "mykey"
		}
		\end{verbatim}
	}
	{
		\color{OliveGreen}
		\begin{verbatim}
		{
		  "section": "hashlist",
		  "request": "getCracked",
		  "response": "OK",
		  "cracked": [
		    {
		      "hash": "098f6bcd4621d373cade4e832627b4f6",
		      "plain": "test",
		      "crackpos": "634721"
		    },
		    {
		      "hash": "5f4dcc3b5aa765d61d8327deb882cf99",
		      "plain": "password",
		      "crackpos": "608529"
		    }
		  ]
		}
		\end{verbatim}
	}






 
	\pagebreak
	\section*{Superhashlists (\textit{superhashlist})}
	Used to access all functions around superhashlists.
	
\subsection*{\textit{listSuperhashlists}}
	List all superhashlists on the server.
	{
		\color{blue}
		\begin{verbatim}
		{
		  "section": "superhashlist",
		  "request": "listSuperhashlists",
		  "accessKey": "mykey"
		}
		\end{verbatim}
	}
	{
		\color{OliveGreen}
		\begin{verbatim}
		{
		  "section": "superhashlist",
		  "request": "listSuperhashlists",
		  "response": "OK",
		  "superhashlists": [
		    {
		      "hashlistId": 2,
		      "hashtypeId": 0,
		      "name": "Test superhashlist",
		      "hashCount": 6494
		    }
		  ]
		}
		\end{verbatim}
	}
\subsection*{\textit{getSuperhashlist}}
	Get detailed information about a superhashlist.
	{
		\color{blue}
		\begin{verbatim}
		{
		  "section": "superhashlist",
		  "request": "getSuperhashlist",
		  "superhashlistId": 2,
		  "accessKey": "mykey"
		}
		\end{verbatim}
	}
	{
		\color{OliveGreen}
		\begin{verbatim}
		{
		  "section": "superhashlist",
		  "request": "getSuperhashlist",
		  "response": "OK",
		  "hashlistId": 2,
		  "hashtypeId": 0,
		  "name": "Test superhashlist",
		  "hashCount": 6494,
		  "cracked": 80,
		  "accessGroupId": 1,
		  "isSecret": false,
		  "hashlists": [
		    1
		  ]
		}
		\end{verbatim}
	}
\subsection*{\textit{createSuperhashlist}}
	Create a new superhashlist out of existing hashlists.
	{
		\color{blue}
		\begin{verbatim}
		{
		  "section": "superhashlist",
		  "request": "createSuperhashlist",
		  "name": "New Superhashlist",
		  "hashlists": [
		    1,
		    2
		  ],
		  "accessKey": "mykey"
		}
		\end{verbatim}
	}
	{
		\color{OliveGreen}
		\begin{verbatim}
		{
		  "section": "superhashlist",
		  "request": "createSuperhashlist",
		  "response": "OK"
		}
		\end{verbatim}
	}
\subsection*{\textit{deleteSuperhashlist}}
	Deletes a superhashlist. But the containing hashlists will not be removed.
	{
		\color{blue}
		\begin{verbatim}
		{
		  "section": "superhashlist",
		  "request": "deleteSuperhashlist",
		  "superhashlistId": 6,
		  "accessKey": "mykey"
		}
		\end{verbatim}
	}
	{
		\color{OliveGreen}
		\begin{verbatim}
		{
		  "section": "superhashlist",
		  "request": "deleteSuperhashlist",
		  "response": "OK"
		}
		\end{verbatim}
	}








	\pagebreak
	\section*{Files (\textit{file})}
	Used to access all functions around files. There are currently following file types available:
	\begin{description}
		\item[0] Wordlist
		\item[1] Rule
	\end{description}
	
\subsection*{\textit{listFiles}}
	List all available files.
	{
		\color{blue}
		\begin{verbatim}
		{
		  "section": "file",
		  "request": "listFiles",
		  "accessKey": "mykey"
		}
		\end{verbatim}
	}
	{
		\color{OliveGreen}
		\begin{verbatim}
		{
		  "section": "file",
		  "request": "listFiles",
		  "response": "OK",
		  "files": [
		    {
		      "fileId": 1,
		      "fileType": 0,
		      "filename": "example.dict"
		    },
		    {
		      "fileId": 2,
		      "fileType": 1,
		      "filename": "dive.rule"
		    },
		    {
		      "fileId": 3,
		      "fileType": 1,
		      "filename": "generated.rule"
		    },
		    {
		      "fileId": 3653,
		      "fileType": 0,
		      "filename": "top10000.txt"
		    },
		    {
		      "fileId": 3654,
		      "fileType": 1,
		      "filename": "toggles4.rule"
		    }
		  ]
		}
		\end{verbatim}
	}
\subsection*{\textit{getFile}}
	Get detailed informations of a file and also get a link to download it.
	{
		\color{blue}
		\begin{verbatim}
		{
		  "section": "file",
		  "request": "getFile",
		  "fileId": 1,
		  "accessKey": "mykey"
		}
		\end{verbatim}
	}
	{
		\color{OliveGreen}
		\begin{verbatim}
		{
		  "section": "file",
		  "request": "getFile",
		  "response": "OK",
		  "fileId": 1,
		  "fileType": 0,
		  "filename": "example.dict",
		  "isSecret": true,
		  "size": 1080240,
		  "url": "getFile.php?file=1&apiKey=mykey"
		}
		\end{verbatim}
	}
\subsection*{\textit{renameFile}}
	Rename an existing file. ATTENTION: this can affect tasks using this file! It is only recommended to change the name when the file is not used by any task!
	{
		\color{blue}
		\begin{verbatim}
		{
		  "section": "file",
		  "request": "renameFile",
		  "fileId": 1,
		  "filename": "example.txt",
		  "accessKey": "mykey"
		}
		\end{verbatim}
	}
	{
		\color{OliveGreen}
		\begin{verbatim}
		{
		  "section": "file",
		  "request": "renameFile",
		  "response": "OK"
		}
		\end{verbatim}
	}
\subsection*{\textit{setSecret}}
	Set if an existing file is secret or not.
	{
		\color{blue}
		\begin{verbatim}
		{
		  "section": "file",
		  "request": "setSecret",
		  "fileId": 1,
		  "isSecret": false,
		  "accessKey": "mykey"
		}
		\end{verbatim}
	}
	{
		\color{OliveGreen}
		\begin{verbatim}
		{
		  "section": "file",
		  "request": "setSecret",
		  "response": "OK"
		}
		\end{verbatim}
	}
\subsection*{\textit{deleteFile}}
	Deletes a file from the server. This is only possible if the file is not used in any task.
	{
		\color{blue}
		\begin{verbatim}
		{
		  "section": "file",
		  "request": "deleteFile",
		  "fileId": 3654,
		  "accessKey": "mykey"
		}
		\end{verbatim}
	}
	{
		\color{OliveGreen}
		\begin{verbatim}
		{
		  "section": "file",
		  "request": "deleteFile",
		  "response": "OK"
		}
		\end{verbatim}
	}
\subsection*{\textit{setFileType}}
	Switch the file type of an existing file.
	{
		\color{blue}
		\begin{verbatim}
		{
		  "section": "file",
		  "request": "setFileType",
		  "fileId": 1,
		  "fileType": 1,
		  "accessKey": "mykey"
		}
		\end{verbatim}
	}
	{
		\color{OliveGreen}
		\begin{verbatim}
		{
		  "section": "file",
		  "request": "setFileType",
		  "response": "OK"
		}
		\end{verbatim}
	}
\subsection*{\textit{addFile}}
	There are multiple ways to add a file, either from an URL, from the import directory or inline. The filename is only relevant if it is added inline. Otherwise it will take the original filename. In case of the inline upload, the data must be base64 encoded (using UTF-8).
	{
		\color{blue}
		\begin{verbatim}
		{
		  "section": "file",
		  "request": "addFile",
		  "filename": "api_test_inline.txt",
		  "fileType": 0,
		  "source": "inline",
		  "accessGroupId": 1,
		  "data": "MTIzNA0KNTY3OA0KcGFzc3dvcmQNCmFiYw==",
		  "accessKey": "mykey"
		}
		\end{verbatim}
	}
	{
		\color{blue}
		\begin{verbatim}
		{
		  "section": "file",
		  "request": "addFile",
		  "filename": "doesnt-matter.txt",
		  "fileType": 0,
		  "source": "import",
		  "accessGroupId": 1,
		  "data": "otherlist.txt",
		  "accessKey": "mykey"
		}
		\end{verbatim}
	}
	{
		\color{blue}
		\begin{verbatim}
		{
		  "section": "file",
		  "request": "addFile",
		  "filename": "doesnt-matter.txt",
		  "fileType": 1,
		  "source": "url",
		  "accessGroupId": 1,
		  "data": "https://github.com/hashcat/hashcat/raw/master/rules/best64.rule",
		  "accessKey": "mykey"
		}
		\end{verbatim}
	}
	{
		\color{OliveGreen}
		\begin{verbatim}
		{
		  "section": "file",
		  "request": "addFile",
		  "response": "OK"
		}
		\end{verbatim}
	}









	\pagebreak
	\section*{Crackers (\textit{cracker})}
	Used to access all functions around crackers and their versions.
	
\subsection*{\textit{listCrackers}}
	Lists all different types of crackers available.
	{
		\color{blue}
		\begin{verbatim}
		{
		  "section": "cracker",
		  "request": "listCrackers",
		  "accessKey": "mykey"
		}
		\end{verbatim}
	}
	{
		\color{OliveGreen}
		\begin{verbatim}
		{
		  "section": "cracker",
		  "request": "listCrackers",
		  "response": "OK",
		  "crackers": [
		    {
		      "crackerTypeId": 1,
		      "crackerTypeName": "hashcat"
		    }
		  ]
		}
		\end{verbatim}
	}
\subsection*{\textit{getCracker}}
	Get detailed informations of cracker, especially all available versions.
	{
		\color{blue}
		\begin{verbatim}
		{
		  "section": "cracker",
		  "request": "getCracker",
		  "crackerTypeId": 1,
		  "accessKey": "mykey"
		}
		\end{verbatim}
	}
	{
		\color{OliveGreen}
		\begin{verbatim}
		{
		  "section": "cracker",
		  "request": "getCracker",
		  "response": "OK",
		  "crackerTypeId": 1,
		  "crackerTypeName": "hashcat",
		  "crackerVersions": [
		    {
		      "versionId": 1,
		      "version": "4.1.0",
		      "downloadUrl": "https:\/\/hashcat.net\/files\/hashcat-4.1.0.7z",
		      "binaryBasename": "hashcat"
		    },
		    {
		      "versionId": 3,
		      "version": "4.1.1",
		      "downloadUrl": "https:\/\/hashcat.net\/beta\/hashcat-4.1.1-15.7z",
		      "binaryBasename": "hashcat"
		    }
		  ]
		}
		\end{verbatim}
	}
\subsection*{\textit{deleteCracker}}
	Deletes a complete cracker type including all versions configured. This is only possible if the type is not used in any of the preconfigured/normal tasks.
	{
		\color{blue}
		\begin{verbatim}
		{
		  "section": "cracker",
		  "request": "deleteCracker",
		  "crackerTypeId": 1,
		  "accessKey": "mykey"
		}
		\end{verbatim}
	}
	{
		\color{OliveGreen}
		\begin{verbatim}
		{
		  "section": "cracker",
		  "request": "deleteCracker",
		  "response": "OK"
		}
		\end{verbatim}
	}
\subsection*{\textit{deleteVersion}}
	Deletes a specific cracker version. This is only possible if the type is not used in any of the tasks.
	{
		\color{blue}
		\begin{verbatim}
		{
		  "section": "cracker",
		  "request": "deleteVersion",
		  "crackerVersionId": 3,
		  "accessKey": "mykey"
		}
		\end{verbatim}
	}
	{
		\color{OliveGreen}
		\begin{verbatim}
		{
		  "section": "cracker",
		  "request": "deleteVersion",
		  "response": "OK"
		}
		\end{verbatim}
	}
\subsection*{\textit{createCracker}}
	Creates a new cracker type.
	{
		\color{blue}
		\begin{verbatim}
		{
		  "section": "cracker",
		  "request": "createCracker",
		  "crackerName": "My Generic Cracker",
		  "accessKey": "mykey"
		}
		\end{verbatim}
	}
	{
		\color{OliveGreen}
		\begin{verbatim}
		{
		  "section": "cracker",
		  "request": "createCracker",
		  "response": "OK"
		}
		\end{verbatim}
	}
\subsection*{\textit{addVersion}}
	Add a new version to an existing cracker type.
	{
		\color{blue}
		\begin{verbatim}
		{
		  "section": "cracker",
		  "request": "addVersion",
		  "crackerTypeId": 2,
		  "crackerBinaryVersion": "1.0.0",
		  "crackerBinaryBasename": "cracker",
		  "crackerBinaryUrl": "https://example.org/download.7z",
		  "accessKey": "mykey"
		}
		\end{verbatim}
	}
	{
		\color{OliveGreen}
		\begin{verbatim}
		{
		  "section": "cracker",
		  "request": "addVersion",
		  "response": "OK"
		}
		\end{verbatim}
	}
\subsection*{\textit{updateVersion}}
	Update the data for an existing cracker version. All values need to be provided, but they do not have to change all.
	{
		\color{blue}
		\begin{verbatim}
		{
		  "section": "cracker",
		  "request": "updateVersion",
		  "crackerVersionId": 4,
		  "crackerBinaryVersion": "1.0.0",
		  "crackerBinaryBasename": "cracker",
		  "crackerBinaryUrl": "https://example.org/archive/download.7z",
		  "accessKey": "mykey"
		}
		\end{verbatim}
	}
	{
		\color{OliveGreen}
		\begin{verbatim}
		{
		  "section": "cracker",
		  "request": "updateVersion",
		  "response": "OK"
		}
		\end{verbatim}
	}







	\pagebreak
	\section*{Server Config (\textit{config})}
	Used to access server configuration.

\subsection*{\textit{listSections}}
	Lists the available config sections.
	{
		\color{blue}
		\begin{verbatim}
		{
		  "section": "config",
		  "request": "listSections",
		  "accessKey": "mykey"
		}
		\end{verbatim}
	}
	{
		\color{OliveGreen}
		\begin{verbatim}
		{
		  "section": "config",
		  "request": "listSections",
		  "response": "OK",
		  "configSections": [
		    {
		      "configSectionId": 1,
		      "name": "Cracking\/Tasks"
		    },
		    {
		      "configSectionId": 2,
		      "name": "Yubikey"
		    },
		    {
		      "configSectionId": 3,
		      "name": "Finetuning"
		    },
		    {
		      "configSectionId": 4,
		      "name": "UI"
		    },
		    {
		      "configSectionId": 5,
		      "name": "Server"
		    }
		  ]
		}
		\end{verbatim}
	}
\subsection*{\textit{listConfig}}
	List all currently known config values.
	{
		\color{blue}
		\begin{verbatim}
		{
		  "section": "config",
		  "request": "listConfig",
		  "accessKey": "mykey"
		}
		\end{verbatim}
	}
	{
		\color{OliveGreen}
		\begin{verbatim}
		{
		  "section": "config",
		  "request": "listConfig",
		  "response": "OK",
		  "items": [
		    {
		      "item": "agenttimeout",
		      "configSectionId": "1",
		      "itemDescription": "Time in seconds the server will consider a client inactive or timed out."
		    },
            ...
            ...
		    {
		      "item": "benchtime",
		      "configSectionId": "1",
		      "itemDescription": "Time in seconds an agent should benchmark a task"
		    }
		  ]
		}
		\end{verbatim}
	}
\subsection*{\textit{getConfig}}
	Get the type and specific value of a config item. The following config types exist:
    \begin{description}
        \item[string] basically everything is accepted as string
        \item[email] similar to string, except that it's tested if it is a valid email
        \item[number] all kind of numeric value
        \item[checkbox] boolean value (true or false)
    \end{description}
	{
		\color{blue}
		\begin{verbatim}
{
  "section": "config",
  "request": "getConfig",
  "configItem": "hashlistAlias",
  "accessKey": "mykey"
}
		\end{verbatim}
	}
	{
		\color{OliveGreen}
		\begin{verbatim}
{
  "section": "config",
  "request": "getConfig",
  "response": "OK",
  "item": "hashlistAlias",
  "configType": "string",
  "value": "#HL#"
}
		\end{verbatim}
	}
\subsection*{\textit{setConfig}}
	Update a value for a config item. Force only needs to be set to true if a new item should be created. Otherwise it will throw an error when it detects an unknown item.
	{
		\color{blue}
		\begin{verbatim}
{
  "section": "config",
  "request": "setConfig",
  "configItem": "contact",
  "value": "test@example.org",
  "force": false,
  "accessKey": "mykey"
}
		\end{verbatim}
	}
	{
		\color{OliveGreen}
		\begin{verbatim}
{
  "section": "config",
  "request": "setConfig",
  "response": "ERROR",
  "message": "Unknown config item!"
}
		\end{verbatim}
	}
    {
        \color{blue}
        \begin{verbatim}
{
  "section": "config",
  "request": "setConfig",
  "configItem": "contactEmail",
  "value": "test@example.org",
  "force": false,
  "accessKey": "mykey"
}
        \end{verbatim}
    }
    {
        \color{OliveGreen}
        \begin{verbatim}
{
  "section": "config",
  "request": "setConfig",
  "response": "OK"
}
        \end{verbatim}
    }









	\pagebreak
    \section*{User Management (\textit{user})}
	Used to access user management.

\subsection*{\textit{listUsers}}
	List all users on the server.
	{
		\color{blue}
		\begin{verbatim}
{
  "section": "user",
  "request": "listUsers",
  "accessKey": "mykey"
}
		\end{verbatim}
	}
	{
		\color{OliveGreen}
		\begin{verbatim}
{
  "section": "user",
  "request": "listUsers",
  "response": "OK",
  "users": [
    {
      "userId": 1,
      "username": "htp"
    },
    {
      "userId": 2,
      "username": "testuser"
    }
  ]
}
		\end{verbatim}
	}
\subsection*{\textit{getUser}}
	Get detail information about a user account.
	{
		\color{blue}
		\begin{verbatim}
{
  "section": "user",
  "request": "getUser",
  "userId": 2,
  "accessKey": "mykey"
}
		\end{verbatim}
	}
	{
		\color{OliveGreen}
		\begin{verbatim}
{
  "section": "user",
  "request": "getUser",
  "response": "OK",
  "userId": 2,
  "username": "testuser",
  "email": "place@hold.er",
  "rightGroupId": 6,
  "registered": 1530788681,
  "lastLogin": 1531740557,
  "isValid": true,
  "sessionLifetime": 3600
}
		\end{verbatim}
	}
\subsection*{\textit{createUser}}
	Create a new user account (Please note that the server will try to send an email to this address and an error might occur when no email sending is configured!).
	{
		\color{blue}
		\begin{verbatim}
{
  "section": "user",
  "request": "createUser",
  "username": "blabla",
  "email": "admin@example.org",
  "rightGroupId": 6,
  "accessKey": "mykey"
}
		\end{verbatim}
	}
	{
		\color{OliveGreen}
		\begin{verbatim}
{
  "section": "user",
  "request": "createUser",
  "response": "OK"
}
		\end{verbatim}
	}
\subsection*{\textit{disableUser}}
	Disable an account, this will throw out the user of all open session and block any further login requests.
	{
		\color{blue}
		\begin{verbatim}
{
  "section": "user",
  "request": "disableUser",
  "userId": 3,
  "accessKey": "mykey"
}
		\end{verbatim}
	}
	{
		\color{OliveGreen}
		\begin{verbatim}
{
  "section": "user",
  "request": "disableUser",
  "response": "OK"
}
		\end{verbatim}
	}
\subsection*{\textit{enableUser}}
	Enables a user account so it can log in.
	{
		\color{blue}
		\begin{verbatim}
{
  "section": "user",
  "request": "enableUser",
  "userId": 3,
  "accessKey": "mykey"
}
		\end{verbatim}
	}
	{
		\color{OliveGreen}
		\begin{verbatim}
{
  "section": "user",
  "request": "enableUser",
  "response": "OK"
}
		\end{verbatim}
	}
\subsection*{\textit{disableLDAP}}
	Disables LDAP authentication for an account.
	{
		\color{blue}
		\begin{verbatim}
{
  "section": "user",
  "request": "disableLDAP",
  "userId": 3,
  "accessKey": "mykey"
}
		\end{verbatim}
	}
	{
		\color{OliveGreen}
		\begin{verbatim}
{
  "section": "user",
  "request": "disableLDAP",
  "response": "OK"
}
		\end{verbatim}
	}
\subsection*{\textit{enableLDAP}}
	Enables LDAP authentication for an account.
	{
		\color{blue}
		\begin{verbatim}
{
  "section": "user",
  "request": "enableLDAP",
  "userId": 3,
  "accessKey": "mykey"
}
		\end{verbatim}
	}
	{
		\color{OliveGreen}
		\begin{verbatim}
{
  "section": "user",
  "request": "enableLDAP",
  "response": "OK"
}
		\end{verbatim}
	}
\subsection*{\textit{setUserPassword}}
	Set a new password for the user. This will not affect open sessions, only new logins.
	{
		\color{blue}
		\begin{verbatim}
{
  "section": "user",
  "request": "setUserPassword",
  "userId": 3,
  "password": "password123",
  "accessKey": "mykey"
}
		\end{verbatim}
	}
	{
		\color{OliveGreen}
		\begin{verbatim}
{
  "section": "user",
  "request": "setUserPassword",
  "response": "OK"
}
		\end{verbatim}
	}
\subsection*{\textit{setUserRightGroup}}
	Assigns the user to the specified right group regarding what it is allowed to access.
	{
		\color{blue}
		\begin{verbatim}
{
  "section": "user",
  "request": "setUserRightGroup",
  "userId": 3,
  "rightGroupId": 6,
  "accessKey": "mykey"
}
		\end{verbatim}
	}
	{
		\color{OliveGreen}
		\begin{verbatim}
{
  "section": "user",
  "request": "setUserRightGroup",
  "response": "OK"
}
		\end{verbatim}
	}




	\pagebreak
    \section*{Group Management (\textit{group})}
	Used to access group management.

\subsection*{\textit{listGroups}}
	List all existing groups.
	{
		\color{blue}
		\begin{verbatim}
{
  "section": "group",
  "request": "listGroups",
  "accessKey": "mykey"
}
		\end{verbatim}
	}
	{
		\color{OliveGreen}
		\begin{verbatim}
{
  "section": "group",
  "request": "listGroups",
  "response": "OK",
  "groups": [
    {
      "groupId": 1,
      "name": "Default Group"
    }
  ]
}
		\end{verbatim}
	}
\subsection*{\textit{getGroup}}
	Get details about a group and its members.
	{
		\color{blue}
		\begin{verbatim}
{
  "section": "group",
  "request": "getGroup",
  "groupId": 1,
  "accessKey": "mykey"
}
		\end{verbatim}
	}
	{
		\color{OliveGreen}
		\begin{verbatim}
{
  "section": "group",
  "request": "getGroup",
  "response": "OK",
  "groupId": 1,
  "name": "Default Group",
  "users": [
    1,
    2,
    3
  ],
  "agents": [
    2,
    3
  ]
}
		\end{verbatim}
	}
\subsection*{\textit{createGroup}}
	Create a new empty group.
	{
		\color{blue}
		\begin{verbatim}
{
  "section": "group",
  "request": "createGroup",
  "name": "Secret Group",
  "accessKey": "mykey"
}
		\end{verbatim}
	}
	{
		\color{OliveGreen}
		\begin{verbatim}
{
  "section": "group",
  "request": "createGroup",
  "response": "OK"
}
		\end{verbatim}
	}
\subsection*{\textit{addAgent}}
	Add agent as a member of a group.
	{
		\color{blue}
		\begin{verbatim}
{
  "section": "group",
  "request": "addAgent",
  "groupId": 3,
  "agentId": 2,
  "accessKey": "mykey"
}
		\end{verbatim}
	}
	{
		\color{OliveGreen}
		\begin{verbatim}
{
  "section": "group",
  "request": "addAgent",
  "response": "OK"
}
		\end{verbatim}
	}
\subsection*{\textit{removeAgent}}
	Remove an agent as member of a group.
	{
		\color{blue}
		\begin{verbatim}
{
  "section": "group",
  "request": "removeAgent",
  "groupId": 3,
  "agentId": 2,
  "accessKey": "mykey"
}
		\end{verbatim}
	}
	{
		\color{OliveGreen}
		\begin{verbatim}
{
  "section": "group",
  "request": "removeAgent",
  "response": "OK"
}
		\end{verbatim}
	}
\subsection*{\textit{addUser}}
	Add a user as member of a group.
	{
		\color{blue}
		\begin{verbatim}
{
  "section": "group",
  "request": "addUser",
  "groupId": 3,
  "userId": 1,
  "accessKey": "mykey"
}
		\end{verbatim}
	}
	{
		\color{OliveGreen}
		\begin{verbatim}
{
  "section": "group",
  "request": "addUser",
  "response": "OK"
}
		\end{verbatim}
	}
\subsection*{\textit{removeUser}}
	Remove a user as member of a group.
	{
		\color{blue}
		\begin{verbatim}
{
  "section": "group",
  "request": "removeUser",
  "groupId": 3,
  "userId": 1,
  "accessKey": "mykey"
}
		\end{verbatim}
	}
	{
		\color{OliveGreen}
		\begin{verbatim}
{
  "section": "group",
  "request": "removeUser",
  "response": "OK"
}
		\end{verbatim}
	}
\subsection*{\textit{deleteGroup}}
	Deletes a group and its associations to users and agents. Please note that all hashlists/tasks belonging to this group will be assigned to the default group.
	{
		\color{blue}
		\begin{verbatim}
{
  "section": "group",
  "request": "deleteGroup",
  "groupId": 3,
  "accessKey": "mykey"
}
		\end{verbatim}
	}
	{
		\color{OliveGreen}
		\begin{verbatim}
{
  "section": "group",
  "request": "deleteGroup",
  "response": "OK"
}
		\end{verbatim}
	}
\subsection*{\textit{abortChunksGroup}}
The client requests chunks to be aborted for agents belonging to a specific access group.
{
	\color{blue}
	\begin{verbatim}
{
  "section": "group",
  "request": "abortChunksGroup",
  "groupId": 1,
  "accessKey": "myKey"
}
	\end{verbatim}
}
{
	\color{OliveGreen}
	\begin{verbatim}
{
  "section": "group",
  "request": "abortChunksGroup",
  "response": "OK"
}
	\end{verbatim}
}










    \pagebreak
    \section*{Access Management (\textit{access})}
	Used to manage permissions for user groups.

\subsection*{\textit{listGroups}}
	List all user groups.
	{
		\color{blue}
		\begin{verbatim}
{
  "section": "access",
  "request": "listGroups",
  "accessKey": "mykey"
}
		\end{verbatim}
	}
	{
		\color{OliveGreen}
		\begin{verbatim}
{
  "section": "access",
  "request": "listGroups",
  "response": "OK",
  "rightGroups": [
    {
      "rightGroupId": 1,
      "name": "Administrator"
    },
    {
      "rightGroupId": 6,
      "name": "Testing Group"
    }
  ]
}
		\end{verbatim}
	}
\subsection*{\textit{getGroup}}
	Get detailed information about a group. If it is the administrators group, the permissions will just be set to \textit{ALL}.
	{
		\color{blue}
		\begin{verbatim}
{
  "section": "access",
  "request": "getGroup",
  "rightGroupId": 6,
  "accessKey": "mykey"
}
		\end{verbatim}
	}
	{
		\color{OliveGreen}
		\begin{verbatim}
{
  "section": "access",
  "request": "getGroup",
  "response": "OK",
  "rightGroupId": 6,
  "name": "Testing Group",
  "permissions": {
    "viewHashlistAccess": false,
    "manageHashlistAccess": false,
    "createHashlistAccess": false,
    "createSuperhashlistAccess": false,
    "viewHashesAccess": false,
    "viewAgentsAccess": false,
    "manageAgentAccess": false,
    "createAgentAccess": false,
    "viewTaskAccess": false,
    "runTaskAccess": true,
    "createTaskAccess": false,
    "manageTaskAccess": false,
    "viewPretaskAccess": true,
    "createPretaskAccess": false,
    "managePretaskAccess": false,
    "viewSupertaskAccess": false,
    "createSupertaskAccess": false,
    "manageSupertaskAccess": false,
    "viewFileAccess": false,
    "manageFileAccess": false,
    "addFileAccess": false,
    "crackerBinaryAccess": false,
    "serverConfigAccess": false,
    "userConfigAccess": false,
	"manageAccessGroupAccess": false
  },
  "members": [
    2,
    3
  ]
}
		\end{verbatim}
	}
\subsection*{\textit{createGroup}}
	Creates a new group.
	{
		\color{blue}
		\begin{verbatim}
{
  "section": "access",
  "request": "createGroup",
  "name": "read only users",
  "accessKey": "mykey"
}
		\end{verbatim}
	}
	{
		\color{OliveGreen}
		\begin{verbatim}
{
  "section": "access",
  "request": "createGroup",
  "response": "OK"
}
		\end{verbatim}
	}
\subsection*{\textit{deleteGroup}}
	Delete a group. This is only possible if no user is associated with this group.
	{
		\color{blue}
		\begin{verbatim}
{
  "section": "access",
  "request": "deleteGroup",
  "rightGroupId": 7,
  "accessKey": "mykey"
}
		\end{verbatim}
	}
	{
		\color{OliveGreen}
		\begin{verbatim}
{
  "section": "access",
  "request": "deleteGroup",
  "response": "OK"
}
		\end{verbatim}
	}
\subsection*{\textit{setPermissions}}
	Change some permissions on this group. Note that all the permissions which are not included in the request will be set to false.
	{
		\color{blue}
		\begin{verbatim}
{
  "section": "access",
  "request": "setPermissions",
  "rightGroupId": 6,
  "permissions": {
    "viewHashlistAccess": true
  },
  "accessKey": "mykey"
}
		\end{verbatim}
	}
	{
		\color{OliveGreen}
		\begin{verbatim}
{
  "section": "access",
  "request": "setPermissions",
  "response": "OK"
}
		\end{verbatim}
	}



    \pagebreak
    \section*{Account (\textit{account})}
	Manage the own account, associated with the API key.

\subsection*{\textit{getInformation}}
	Get general information about the account.
	{
		\color{blue}
		\begin{verbatim}
{
  "section": "account",
  "request": "getInformation",
  "accessKey": "mykey"
}
		\end{verbatim}
	}
	{
		\color{OliveGreen}
		\begin{verbatim}
{
  "section": "account",
  "request": "getInformation",
  "response": "OK",
  "userId": 1,
  "email": "htp@htp.htp",
  "rightGroupId": 1,
  "sessionLength": 36000
}
		\end{verbatim}
	}
\subsection*{\textit{setEmail}}
	Update the email of the user.
	{
		\color{blue}
		\begin{verbatim}
{
  "section": "account",
  "request": "setEmail",
  "email": "place@hold.er",
  "accessKey": "mykey"
}
		\end{verbatim}
	}
	{
		\color{OliveGreen}
		\begin{verbatim}
{
  "section": "account",
  "request": "setEmail",
  "response": "OK"
}
		\end{verbatim}
	}
\subsection*{\textit{setSessionLength}}
	Set the time in seconds after a session of this user times out.
	{
		\color{blue}
		\begin{verbatim}
{
  "section": "account",
  "request": "setSessionLength",
  "sessionLength": 6000,
  "accessKey": "mykey"
}
		\end{verbatim}
	}
	{
		\color{OliveGreen}
		\begin{verbatim}
{
  "section": "account",
  "request": "setSessionLength",
  "response": "OK"
}
		\end{verbatim}
	}
\subsection*{\textit{changePassword}}
	Change the password of the user account.
	{
		\color{blue}
		\begin{verbatim}
{
  "section": "account",
  "request": "changePassword",
  "newPassword": "password",
  "oldPassword": "htp",
  "accessKey": "mykey"
}
		\end{verbatim}
	}
	{
		\color{OliveGreen}
		\begin{verbatim}
{
  "section": "account",
  "request": "changePassword",
  "response": "OK"
}
		\end{verbatim}
	}







		
%\subsection*{\textit{temporary}}
%	TODO
%	{
%		\color{blue}
%		\begin{verbatim}
%		{
%		"section": "agent",
%		"request": "deleteVoucher",
%		"voucher": "Gjawgidkr4",
%		"accessKey": "mykey"
%		}
%		\end{verbatim}
%	}
%	{
%		\color{OliveGreen}
%		\begin{verbatim}
%		{
%		"section": "agent",
%		"request": "deleteVoucher",
%		"response": "OK"
%		}
%		\end{verbatim}
%	}
\end{document}